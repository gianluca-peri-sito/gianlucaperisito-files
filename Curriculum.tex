\documentclass[11pt, a4paper]{moderncv}

% moderncv themes
\moderncvtheme[purple]{classic}

% custom color purple
\definecolor{mcc}{rgb}{0.54,0.17,0.89}


% character encoding
\usepackage[utf8]{inputenc}

% language
\usepackage[italian]{babel}

% Set font
\usepackage{lmodern}

\firstname{Gianluca}
\familyname{Peri}
\title{Curriculum Vitae}
\email{gianluca.peri@unifi.it}
\social[github]{gianluca-peri}
\homepage{www.gianlucaperi.com}

\begin{document}

\makecvtitle

%------------------------

% Education Section
\section*{Istruzione e Formazione}

\cventry{2024-presente}{Dottorato in Machine Learning e Sistemi Complessi}{Università degli Studi di Firenze}{}{}{
\textcolor{mcc}{Supervisor:} Prof. Duccio Fanelli\\
\textcolor{mcc}{Co-Supervisor:} Prof. Francesco Piazza\\
\textcolor{mcc}{In collaborazione con:} Tecnolink S.r.l.\\
\textcolor{mcc}{Research Group:} CSDC: Centro Studio Dinamiche Complesse\\
}

\cventry{2025-present}{Attività di Ricerca}{Università di Namur}{Belgio}{}{
    \textcolor{mcc}{Supervisore:} Prof. Timoteo Carletti\\
    \textcolor{mcc}{Progetto:} Ipergrafi Neurali \& Cammini Aleatori su Loss Landscapes\\
}

\cventry{2021-2024}{Laurea Magistrale in Fisica Teorica}{Università degli Studi di Firenze}{}{}{
\textcolor{mcc}{Indirizzo:} Fisica Teorica dei Sistemi Complessi\\
\textcolor{mcc}{Relatore:} Prof. Duccio Fanelli\\
\textcolor{mcc}{Tesi:} \emph{Reti Neurali con Formalismo Spettrale Generalizzato}\\
\textcolor{mcc}{Voto di laurea:} 110/110\\
}

\cventry{2016-2021}{Laurea Triennale in Fisica e Astrofisica}{Università degli Studi di Firenze}{}{}{
\textcolor{mcc}{Relatore:} Prof. Simone Landi\\
\textcolor{mcc}{Tesi:} \emph{Modelli Fluidi di Venti Supersonici da Corone Stellari}\\
}

\cventry{2011-2016}{Diploma Scientifico}{Liceo Scientifico A. M. Enriques Agnoletti}{Sesto Fiorentino}{}{}

% Employment Section
\section*{Esperienze Lavorative}

\cventry{2025-presente}{Divulgatore Scientifico}{OpenLab - Centro di Servizi per l'Educazione e la Divulgazione Scientifica}{Sesto Fiorentino}{}{}

\cventry{2025-presente}{Tutor Didattico}{Università degli Studi di Firenze}{Firenze}{}{}

\cventry{2021-2024}{Insegnante Liceo Privato}{Istituto Giovanni Dupré}{Firenze}{}{}

\cventry{2018-2019}{Relazioni Internazionali}{Dalle Nostre Mani}{Firenze}{}{}

\newpage

\section{Pubblicazioni}
\cventry{2025}{G. Peri, L. Chicchi, D. Fanelli, L. Giambagli}{SPectral ARchiteCture Search for neural network models}{npj Artificial Intelligence}{https://doi.org/10.1038/s44387-025-00039-1}{}

% Conference Activity Section 
\section*{Contributi a Conferenze e Workshop}

\cventry{12/2025}{\textbf{Poster}}{BeNet 2025}{11th Edition of the Belgian Network Research Meeting}{Ghent}{\textcolor{mcc}{Titolo del contributo:} \emph{Spectral Architecture Search for Neural Networks}}

\cventry{07/2025}{\textbf{Talk}}{Nordita}{Fluctuations in Self-Interacting and Learning Processes}{Stoccolma}{
  \textcolor{mcc}{Titolo contributo:} \emph{Training Walkers with Reinforcement Learning}
}

\cventry{0/2025}{\textbf{Poster}}{StatPhys29}{29th International Conference on Statistical Physics}{Firenze}{
  \textcolor{mcc}{Titolo contributo:} \emph{Spectral Architecture Search for Neural Networks}
}
  \cventry{07/2024}{\textbf{Talk}}{COMPENG 2024}{IEEE Workshop on Complexity in Engineering}{Firenze}{
    \textcolor{mcc}{Titolo contributo:} \emph{Generalized Spectral Formalism for Neural Networks}
}

% Languages Section
\section*{Lingue}

\cvlanguage{Italiano}{Madrelingua}{}
\cvlanguage{Inglese}{Avanzato, livello C1}{}
\cvlanguage{Francese}{Basico, livello A2}{}

% References Section 
\section*{Referenti}

\cvline{Prof. Duccio Fanelli}{Professore Ordinario - \textit{Università degli Studi di Firenze}}
\cvline{Dr. Lorenzo Giambagli}{Postdoc - \textit{Free University of Berlin}}

%------------------------

\vfill

\begin{flushright}
\textsc{aggiornato in data:} \today
\end{flushright}


\end{document}
